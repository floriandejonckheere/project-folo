\documentclass[10pt,a4paper]{article}

\usepackage[utf8]{inputenc}
\usepackage[dutch]{babel}
\usepackage{fancyhdr}
\usepackage{geometry}
\usepackage{graphicx}
\usepackage{tabularx}
\usepackage{wallpaper}
\usepackage{listings}
\usepackage{amsmath}
\usepackage{amssymb}
\usepackage{./alloy}

\usepackage{xcolor, colortbl}
\definecolor{ugentblue}{HTML}{164A7C}
\definecolor{gray}{HTML}{AAAAAA}
\definecolor{lightgray}{HTML}{FAFAFA}
\definecolor{grayborder}{HTML}{CCCCCC}
\definecolor{commentgreen}{HTML}{009900}

% Tables
\def\arraystretch{1.35}
\renewcommand{\tabularxcolumn}[1]{>{\small}m{#1}}
\newcommand{\hcell}[1]{
	\cellcolor{ugentblue}\color{white}\textbf{#1}
}

\makeatletter
\renewcommand\thesubsection{\@arabic\c@section.\@arabic\c@subsection}
\makeatother{}

\usepackage[hypertexnames=false]{hyperref}
\usepackage[numbered, depth=3]{bookmark}

\renewcommand{\headrulewidth}{0pt}
\pagestyle{fancy}
\fancyhf{}

\interfootnotelinepenalty=0

\lstset{
	backgroundcolor=\color{lightgray},
	basicstyle=\footnotesize,
	commentstyle=\color{commentgreen},
	frame=single,
	keywordstyle=\color{blue},
	language=Alloy,
	numbers=left,
	numbersep=5pt,
	numberstyle=\tiny\color{gray},
	rulecolor=\color{black},
	stepnumber=1,
	stringstyle=\color{ugentblue},
	showspaces=false,
	showstringspaces=false
}

\begin{document}
	\begin{titlepage}
		%% Footer
		\thispagestyle{fancy}
		\fancyhf{}
		
		%% Page
		\hfill
		\begin{minipage}[t][0.9\textheight]{0.8\textwidth}
			\noindent
			\includegraphics[width=55px]{ugent-blue.png} \\[-1em]
			\color{ugentblue}
			\makebox[0pt][l]{\rule{1.3\textwidth}{1pt}}
			\par
			\noindent
			\textbf{\textsf{Formele Logica}} \textcolor{gray}{\textsf{Academiejaar 2014-2015}}
			\vfill
				\noindent
			{\huge \textsf{Project Alloy}}
			\vskip\baselineskip
			\noindent
			\textsf{\textbf{Jasper D'haene \textless jasper.dhaene@ugent.be\textgreater} \\
				\textbf{Florian Dejonckheere \textless florian@floriandejonckheere.be\textgreater}}
		\end{minipage}
	\end{titlepage}

	%%% PAGE STYLE %%%
	\nopagecolor
	\renewcommand{\footrulewidth}{0.4pt}
	\headheight 45pt
	\ULCornerWallPaper{1}{header.png}
	\fancyfoot[C]{\thepage}

	%%% DOCUMENT %%%
	\section{Ramsey-getallen}
		Ramsey's theorem dicteert dat in een voldoende grote set waarvan de bogen gekleurd zijn met een willekeurig aantal kleuren, monochromatisch gekleurde subsets te vinden zijn. Stel $m, n \in \mathbb{R^+}$ en twee kleuren $k_i, i \in [0, 1]$, dan definieert Ramsey's theorem $R(m, n, k)$ de ondergrens voor de complete graaf die een subset over tenminste $m$ toppen met kleur $k_0$, of een subset over tenminste $n$ toppen met kleur $k_1$ bevat.\\

		\noindent Informeel is dit probleem (voor $k = 2$) ook wel bekend als het \textit{party problem}: hoe groot is de minimale set van personen die uitgenodigd moeten worden voor een feestje, waarvoor geldt dat ofwel minstens $m$ personen elkaar (mutueel) kennen, ofwel $n$ personen elkaar (mutueel) niet kennen.

		\noindent Het Ramseygetal wordt bepaald door het uitvoeren van \'e\'en of meerdere instanties van het predikaat \textit{Sub\_Graph}. Dit predikaat controleer of er disjuncte subsets te vinden zijn van verschillende kleuren. Als er bij dit predikaat een instantie wordt gevonden, is het Ramseygetal gevonden.
		De parameters van het algoritme zijn in te stellen op de laatste regels. Stel het volgende Ramseygetal:
		\begin{equation}
			R(m, n; k) = v
		\end{equation} \\

		Dan kunnen de programmaparameters als volgt geschreven worden:\\

		\begin{lstlisting}
...

run {
	Sub_Graph[m] or Sub_Graph[n]
} for k Colour, exactly v Node, exactly (v * (v - 1)) Edge
		\end{lstlisting}

		\clearpage{}
		\lstinputlisting[caption=Ramseygetallen Alloy model]{../src/ramsey_numbers.als}

	\clearpage{}
	\section{Cyclische toren van Hanoi}
		De cyclische toren van Hanoi is een variant op de bekende combinatorische puzzel. Buiten de drie hoofdregels geldt de volgende regels ook:
		\begin{enumerate}
			\setcounter{enumi}{3}
			\item Schijven kunnen enkel cyclisch naar rechts opgeschoven worden. Voor drie palen $A, B, C$ is dit dus $A \rightarrow B \rightarrow C \rightarrow A$
		\end{enumerate}

		\noindent Als basis voor het algoritme werd de ge\"implementeerde versie van Ilya Shlyakhter beschouwd. Deze versie wordt bijgeleverd als voorbeeld bij de Alloy Analyzer, maar implementeert de niet-cyclische variant.

		\lstinputlisting[caption=Cyclische toren van Hanoi Alloy model]{../src/cyclic_hanoi.als}

\end{document}
